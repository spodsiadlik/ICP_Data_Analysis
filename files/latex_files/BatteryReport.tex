\documentclass{article}
\usepackage{graphicx}
\usepackage{amsmath}
\usepackage{amssymb}
\usepackage{hyperref}

\author{Simon R. Podsiadlik}
\title{Data Analysis: Battery Discharge}
\date{April 30, 2023}

\begin{document}
\maketitle

\section{Abstract}
For this project, we chose to analyze voltage and current data of a discharging 9-Volt battery that we had gathered from the Microcontrollers course. We first plotted the data to physically see the discharge and compare it to what we expected. Then, we used the data to find the overall capacity of an average 9-Volt battery.

\section{Introduction}
The primary objective of this experiment was to measure the discharge of a 9V battery and use that data to determine its overall capacity. To do this, we used an audio-to-digital converter connected to a voltage divider to read the data into a Raspberry Pi. We took voltage and current measurements every five minutes until the battery was no longer in a usable range (lower than 4.8 Volts). Then, we plotted it on a scatter plot. Additionally, we used our current data and integrated it to find the overall capacity of the battery.
\section{Data}
\section{Visualization}
\section{Results}

\end{document}
